Om de specifieke componenten van het PCB te lezen, zie de Bill of Materials (BOM) en PCB document.\vspace{1em}

Als kabels zijn micro molex kabels gebruikt. deze zijn besteld bij \href{https://eu.mouser.com/ProductDetail/Molex/79758-2140?qs=d0WKAl\%252BL4KbSUmFUgUy5Sg\%3D\%3D}{79758-2140}, dit zijn voorgekrimpte kabels die in de Molex nanoFit header past. De \href{https://nl.mouser.com/ProductDetail/Molex/105308-1204?qs=PqFvFmEiOr5yxx9OZCbUfg\%3D\%3D&countrycode=DE&currencycode=EUR}{105308-1204} is de Molex NanoFit header. Deze wordt gebruikt om de sensoren te kunnen repluggen. deze connectors passen in de \href{https://nl.mouser.com/ProductDetail/Molex/105310-1104?qs=PqFvFmEiOr6mewoWpXKrGw\%3D\%3D}{105310-1104} male connector.\vspace{1em}

Voor het ijken van de dust sensor en de fijnstof sensor zijn referentie apparaten geregeld. De \href{https://www.bol.com/nl/nl/p/airvital-fijnstofmeter-luchtkwaliteitsmeter-pm2-5/9300000024155194/?s2a=}{Airvital fijnstof meter} wordt gebruikt om de fijnstofsensor te kunnen ijken. de \href{https://www.bol.com/nl/nl/p/yilofy-professionele-5-in-1-luchtkwaliteitsmeter-hygrometer-co2-meter-horeca-draagbaar-lcd-scherm-monitor-co2-luchtvochtigheidsmeter-sensor-melder-temperatuur-thermometer-binnen-buiten-oplaadbaar-usb-kabel/9300000054426695/?s2a=}{YILOFY VOC monitor} wordt gebruikt om de voc sensor te kunnen ijken.\vspace{1em}

voor de sensebox zijn ook een aantal onderdelen besteld. Hiervoor zijn 2 SD kaarten voor de logger en de config file besteld. Voor de microSD kaarten maakt het niet uit welke het is, hierom is er gekozen voor een \href{https://www.bol.com/nl/nl/p/micro-sd-kaart-zwart-32gb-adapter-class-10/9200000064927040/?s2a=}{Kingston microSD}. Om de SD kaart te testen voordat de bordjes beschikbaar waren zijn er \href{https://www.amazon.nl/gp/product/B077MCQS9P/ref=ppx_yo_dt_b_asin_title_o08_s00?ie=UTF8&psc=1}{SPI sd reader breakouts van azdelivery} en esp32 devboards besteld, de \href{https://nl.mouser.com/ProductDetail/Espressif-Systems/ESP32-DevKitC-32D?qs=\%252BEew9\%252B0nqrDsObWEpDx6YQ\%3D\%3D}{ESP32-DevKitC-32D} is hiervoor gekozen.\vspace{1em}

Voor de sensebox is ook een ac/dc stroomadapter besteld van \href{https://www.amazon.nl/gp/product/B079P8G5FX/ref=ppx_yo_dt_b_asin_title_o03_s00?ie=UTF8&psc=1}{PChero} die tussen kan worden ingesteld op onderanderen 3v3 of 5 volt met 2A max. Er is ook een \href{https://www.amazon.nl/gp/product/B07N978W56/ref=ppx_yo_dt_b_asin_title_o04_s00?ie=UTF8&psc=1}{CP2104 serial converter} Zodat de sensebox ook nog via uart kan worden gedebugged als er problemen ontstaan met de usb-C verbinding. 